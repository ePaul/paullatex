% \iffalse meta-comment
%
%%  (C) 2008 Paul Ebermann
%%
%%   Package exp-testopt - expandible testopt-Variante.
%%
%%   Die Datei exp-testopt.dtx sowie die dazugehörige
%%   exp-testopt.ins sowie die damit generierte
%%   exp-testopt.sty stehen unter der
%%   "LaTeX Project Public License" (LPPL, zu finden
%%   unter http://www.latex-project.org/lppl/, sowie
%%   auch in den meisten TeX-Distributionen in
%%   texmf/docs/latex/base/lppl*.txt), Version 1.3b oder
%%   später (nach Wahl des Verwenders).
%%
%%   Der 'maintenance-status' ist (zur Zeit) 'author-maintained'.
%%   
%%   Das heißt u.a., die Dateien dürfen frei vertrieben werden,
%%   bei Änderungen (durch andere Personen als Paul Ebermann)
%%   ist aber der Name der Datei zu ändern.
%
% \fi
%
% \iffalse
% ---------------------------------------------
%<package>\NeedsTeXFormat{LaTeX2e}[2003/12/01]%
% ----------------------------________________-----------------
%<package>\ProvidesPackage{exp-testopt}[2009/03/03 v0.2  (PE)]
% ----------------------------________________-----------------
% \fi
%
% \iffalse
%<*driver>
\documentclass{scrartcl}
%
% Esperantajn tekstojn oni prefere entajpu per unikodaj literoj.
\usepackage[utf8x]{inputenc}
\usepackage[esperanto]{babel}

\addto{\extrasesperanto}{%
  \renewcommand*{\generalname}{\^Generale}%
  \GlossaryPrologue{%
    \section{\^San\^goj}%
  }%
  \GlossaryMin=3\baselineskip
  \IndexPrologue{%
    \section{Indekso}
    Kurzivaj nombroj indikas la lokojn, kie la indeks-ero estas priskribita,
    substrekitaj nombroj indikas la lokon de la difino, la aliaj estas uzoj.
  }
  \IndexMin=5\baselineskip
}

\usepackage[svgnames]{xcolor}

\usepackage{exp-testopt}

\usepackage[bookmarks=false]{hyperref}
\usepackage[countalllines, withmarginpar]{gmdoc}
\usepackage[inline, visible]{gmdoc-enhance}
\let\pack\textrm%
\renewcommand*{\EOFMark}{}

\RecordChanges
\begin{document}
   % ^ ne taŭgas kiel specialsigno, ĉar ni
   % bezonas ĝin por ^^B, ^^M ktp.
   \catcode`\^=7%
   \MakeShortVerb\'
   \DocInput{exp-testopt.dtx}
\end{document}
%</driver>
% \fi
%
% \CheckSum{90}
% \chschange{v0.1}{2009/03/03}{37}
% \chschange{v0.2}{2009/03/03}{90}
% \changes{v0.1}{2009/03/03}{Unua versio}
% \changes{v0.2}{2009/03/03}{Preta por publikigo, mi kredas.}
%
% \GetFileInfo{exp-testopt.sty}
%
%
% \title{La \pack{exp-testopt}-Pakaĵo -- ekspandebla varianto de \
%   \cs{@testopt}.  \thanks{ Tiu dokumento rilatas al \
%     \pack{exp-testopt}~\fileversion, de~\filedate.}}
% \author{Paul Ebermann\thanks{\texttt{Paul-Ebermann@gmx.de}}}
%
% \maketitle
%
%
% \begin{abstract}
%   Ekspandebla varianto de '\@testopt' el la \LaTeX-kerno, kaj iom
%   pli ĝenerala (ankaŭ ekspandebla) '\@expandable@ifopt', por rekoni
%   ekziston de opciaj argumentoj (sen tuj doni defaŭlton).
%   Ankaŭ '\newcommand'-varianto, kiu uzas tion.
% \end{abstract}
% 
%
% \tableofcontents
%
% \section{Uzanta dokumentaĵo}
%
%  Tiu ĉi pakaĵo enhavas unu makroon rekte uzebla de uzantoj, kvankam
%  ankaŭ ĝin verŝajne pli ofte uzos pakaĵo-kreantoj.
%  Krome ĝi enhavas du utilajn makroojn por pakaĵ-kreantoj
% (ne por uzantoj).
%
%  \subsection{Makrooj}
%
%  \TextUsage{\expnewcommand}'*'\arg{nomo}\arg[paramnum]\
%   \arg[defaŭlto]\arg{kodo} 
%   --
%   funkcias kiel '\newcommand': Kreas komandon kun \meta{num} parametroj
%   (ĝis 9), kaj se \meta{defaŭlto} estas donita, la unua parametro estos
%   opcia kun \meta{defaŭlto} kiel defaŭlta valoro.
%   Diferenco al '\newcommand' nur ekzistas, se \meta{defaŭlto} estas
%   donita -- tiam je voko de la makroo ni uzas ekspandeblan komparon
%   por ekscii, ĉu unua parametro estas donita. (Tio havas la avantaĝon
%   esti ekspandebla, sed la limigojn menciitajn sube.)
%  
%
%   \TextUsage{\@expandable@testopt}\marg{ordono}\marg{defaŭlto}
%   -- 
%  vokas \meta{ordono}n kun '[]'-parametro -- se neniu plia estas donata de
%  la uzanto, ni uzas \meta{defaŭlto} anstataŭe.
%  Pli detale: kontrolas, ĉu la sekva signo
%  estas '['. Se jes, iĝas \meta{ordono}, alikaze iĝas
%  \meta{ordono}\arg[defaŭlto].
%  (Tio funkcias simile kiel '\@testopt' el la \LaTeX-kerno.)
%
%  \TextUsage{\@expandable@ifopt}\marg{jes}\marg{ne} --
%  kontrolas, ĉu la sekva signo estas '['. Se jes, vokas \meta{jes}, alikaze
%  \meta{ne}.
%
% \subsection{Limigoj}
%
%  Fakte tiuj makrooj ne kontrolas la sekvan simbolon (kio igus ĝin
%   ne-ekspandebla pro uzo de '\futurelet', kiel '\@testopt'),
%  sed la sekvan makroo-argumenton ('#3', fakte). Krome estas malfacile
%  kompari ion, se oni ne povas difini/ŝanĝi ion.
%
% Tio havas la sekvajn konsekvencojn:
% \begin{itemize}
% \item 
%   Ne eblas distingi '[' de '{[}' per nia algoritmo -- ambaŭ estas traktataj
%   kiel \meta{Jes}.
% \item 
%   La sekva argumento estu de la amika speco: Ĝi estu (se ĝi ne estas
%   simple '[') ĉeno el maksimume unu \emph{danĝera} simbolo (kontrolsekvenco
%   aŭ el danĝera kategorio) kaj poste simpla teksto (el kategorioj 10, 11,
%   12).
%
%   Se ĝi konsistas el nur unu simbolo, tio ne estas problemo. Se ĝi
%   konsistas el simpla teksto, tio ankaŭ ne estas problemo.
%
%   En aliaj kazoj povas aperi strangaj erarmesaĝoj.
% \item Apero de pluraj spaco-simboloj antaŭ la '[' povas konfuzigi la
%   programon, tiam ankaŭ aperos strangaj erarmesaĝoj.
% \end{itemize}
%
%  Do, antaŭ la uzo de tiuj makrooj bone pripensu, ĉu la limigoj vin
%  ĝenas.
%
%  (Krome eblas trompi nian algoritmon per kreado de pliaj makrooj en la
%   nomspaco de '\exp-testopt@@'\meta{io}'@'. Ne faru tion, krom se vi certas
%   pri la konsekvencoj.)
%
%
% \section{Ekzemplo}
%
% Jen ekzemplo de la uzo por krei komandon '\beispiel' kun opcia
% argumento.
%
%<*test>
\makeatletter
\errorcontextlines = 20

\def\beispiel{%
  \@expandable@testopt\beispielImpl{Default}%
}
\def\beispielImpl[#1]{%
  \fbox{#1}%
}

  \beispiel{egal} \beispiel[bla]
  \beispiel$ e=m c^2$ \beispiel[$ e=m c^2$]
%</test>
%
% Kontraŭe al kreado de tia komando per '\newcommand', la komando estas
% tute ekspandebla (nu, krom la enhavo de '\beispielImpl'), anstataŭ
% robustigita per malfruigita ekspandado. Tiel ekzemple eblas uzi valorojn
% de variabloj kun ilia aktuala valoro, ankaŭ se la komando mem estos poste
% skribota en dosieron. (Hmm, iom komplika, ĉu?)
%
% La supra kodo donas jenan rezulton:
%
% \normalInput{\jobname.test}
%
% \StopEventually{\PrintChanges\PrintIndex}
%
% \section{Implementado}
%
%
%<*package>

%  La ideo por tia ekspandebla kontrolo de la sekva »signo« estas, ke ni
%  kreas el la »signo« makroonomon, kaj komparas la signifon de ĝi kun
%  signifo de konata makroo.
%
%  \subsection{Komparendaj makrooj}
%
%  Por tio ni kreas iujn nomojn, kiuj estas supozeble ne uzata de iu alia,
%  kun signifoj, kiuj ankaŭ supozeble ne erare aperas.
%
\def\exp@testopt@opt@{<<<[>>>}% tio estas la nomo, kies valoron ni
% komparas kun aliaj kandidatoj. 
\@namedef{exp-testopt@@\string[@}{<<<[>>>}% Tio -- '\exp-testopt@@[@' -- estas
% la makroo, kiu estos trovota en la \emph{jes}-kazo.

%
% \subsection{Ĉefaj makrooj}
%
% Niaj du ĉefaj makrooj nun aspektas tiom similaj, ke ni povas implementi
% 'testopt' per 'ifopt'.
%
\newcommand*{\@expandable@testopt}[2]{%
  \@expandable@ifopt{#1}{#1[#2]}%
}

% '\@expandable@ifopt' nun faras la laboron.
%
\newcommand*{\@expandable@ifopt}[3]{%
  \expandafter\ifx% '\ifx' komparas la difinojn de du makrooj, nome \dots
     \csname exp-testopt@@\string#3@\endcsname% \dots makronomo kreita el la
% tria argumento (= la signo poste).
     \exp@testopt@opt@% \dots kaj tiu antaŭdifinita makroo.
    \afterfi{#1#3}% En la jes-kazo, vokas la unuan argumenton
% (kun la tria), \dots
  \else
    \afterfi{#2#3}% \dots en la ne-kazo la duan.
  \fi%
}


% \subsection{Komando-difinoj}
%
% \changes{v0.2}{2009/03/03}{Aldono de \cs{expnewcommand}.}
%
% Ni volas krei variantojn de '\newcommand' kaj '\renewcommand', kiuj uzas
% nian ekspandeblan metodon por opciaj argumentoj, anstataŭ la robustan.
%
% Jen la originalo el latex.ltx (ltdefns.dtx, por esti preciza).
% 
%<*latex2e>
 \long\def\@xargdef#1[#2][#3]#4{%
   \@ifdefinable#1{% { \color{red}
      \expandafter\def\expandafter#1\expandafter{%
           \expandafter%
           \@protected@testopt%
           \expandafter% 
           #1%
           \csname\string#1\endcsname%
           {#3}}% }
        \expandafter\@yargdef
           \csname\string#1\endcsname
            \tw@
            {#2}%
            {#4}}}
%</latex2e>
%
% Kaj ni nun devas ŝanĝi la markitan parton
% por anstataŭe uzi nian metodon. Tio signifas, ke \meta{nomo} nun estas
% difinita kiel
% {\color{Green}'\@expandable@testopt\'\meta{nomo}\marg{defaŭlto}},
% anstataŭ
% {\color{red}^^A
% '\@protected@testopt'\meta{nomo}'\'\meta{nomo}\marg{defaŭlto}}.
%
% 
\long\def\@exp@xargdef#1[#2][#3]#4{%
  \@ifdefinable#1{% { \color{Green}
    \expandafter\def\expandafter#1\expandafter{% 
      \expandafter%
      \@expandable@testopt%
      \csname\string#1\endcsname%
      {#3}}% }
    \expandafter\@yargdef%
      \csname\string#1\endcsname%
      \tw@%
      {#2}%
      {#4}%
  }%
}


% Kaj nun kelkaj paralelaj difinoj al '\newcommand', '\new@command' kaj
%  '\@newcommand'.
%
\def\expnewcommand{%
  \@star@or@long\expnew@command%
}
\def\exp@new@command#1{%
  \@testopt{\exp@newcommand#1}0%
}
\def\exp@newcommand#1[#2]{%
  \ifnextchar [{\@exp@xargdef#1[#2]}% Ĉi tie ni vokas nian varianton
% anstataŭ '\@xargdef'. ^^A ]
               {\@argdef#1[#2]}% Tiu estas la originalo.
}





%
% \subsection{Helpaj makrooj}
%
% '\afterfi'\arg{kodo} -- saltas ĝis post la sekva '\fi', kaj
% ekzekutas \meta{kodo}n tie.
% 
% Kopiita el \pk{gmutils}.
%
\@ifundefined{afterfi}{%
  \def\afterfi#1#2\fi{%
    \fi#1%
    }%
}{}%

%
%
%
% Jam fino :-)
\endinput
%</package>
% (Jes, vere fino.)
%
% \Finale
%\endinput


%%% Folgendes ist nur für meinen Editor.
%%%
%%% Local Variables:
%%% mode: docTeX
%%% TeX-master: t
%%% End:
